\documentclass[12pt]{article}
%\pagestyle{plain}
%\usepackage[T2A]{fontenc}
\usepackage[utf8]{inputenc}
%\usepackage[russian]{babel}
\usepackage{amsmath, amssymb, amsthm}
\usepackage{latexsym}
%\usepackage[dvips]{graphicx}
\usepackage{amssymb}
\usepackage{graphicx}
%\usepackage{here}
%\usepackage{wrapfig}
 \hoffset=-0.5cm
 \voffset=-2.5cm
 \textwidth=16.5cm
 \textheight=24.0cm

 \begin{document}

 \section{Approximation of a polygonal boundary of a Bezier curve}

 In order to be able to evenly distribute points on the resulting boundary,
 we approximate the boundary with a certain parametric curve $\phi(t), t \in [0, 1]$.
 In addition, this will allow us to smooth the curve from unevenness.
 As such a curve, we will take the Bezier curve \cite{mesteckiy}.
 
 A Bezier curve is a parametric curve of the form: $\sum _{j=0}^{m}V_{j} B_{j}^{m} \left(t\right) $,
 where the parameter $t$ lies in the range $\left[0,\; 1\right]$,
 the ordered set of polygons $V=\left\{V_{0} ,{\kern 1pt} \; V_{1} ,\ldots ,\; V_{m} \right\}$
 is called the characteristic polygon of the Bezier curve,
 and $B_{j}^{m} (t)=\frac{m!}{(m-j)!j!} (1-t)^{m-j} t^{j} ,\; j=0,\ldots ,m$ are the Bernstein polynomials.
 A particular curve from this class is defined by the characteristic polygon.
 Thus, the vector of parameters $\alpha $, defining the curve $!_{\alpha } $,
 in relation to Bezier curves consists of $2\cdot (m+1)$ parameters -- coordinates $V_{1} ,V_{2} ,\ldots V_{m} $.
 
 Let us denote the coordinates of the initial points $P=\{ P_{0} ,\ldots ,\; P_{n} \} $ by $P_{i} =(a_{i} ,\; b_{i} ),\; i=1,\ldots ,\; n$,
 and the coordinates of the sought vertices of the characteristic polygon $V=\left\{V_{0} ,{\kern 1pt} \; V_{1} ,\ldots ,\; V_{m} \right\}$,
 defining the Bezier curve by $V_{j} =(x_{j} ,y_{j} ),\; j=1,\ldots ,m$.
 
 In these notations we obtain the parameter vector $\alpha =(x_{0} ,y_{0} \ldots ,x_{m} ,y_{m} )$. Then the curve $C_{\alpha } $is described as follows:
 
 \begin{equation}
   C(t;x_{0} ,y_{0} ,\ldots ,x_{m} ,y_{m} )=(\sum _{j=0}^{m}x_{j} B_{j}^{m} (t),\sum _{j=0}^{m}y_{j} B_{j}^{m} (t)  ).
 \end{equation}
 
 The sum of the squares of the distances from the points in $P$ to this curve is:
 \begin{eqnarray}
 \label{slabel}
 S(\alpha ) &=& \sum _{i=0}^{n}\left[d\left(P_{i} ,C(t_{i} ;\alpha )\right)\right] ^{2} = \nonumber \\
 &=& \sum _{i=0}^{n}\left[\left(a_{i} -\sum _{j=0}^{m}x_{j} B_{j}^{m} (t_{i} ) \right)^{2} +\left(b_{i} -\sum _{j=0}^{m}y_{j} B_{j}^{m} (t_{i} ) \right)^{2} \right],
 \end{eqnarray}
 
 The problem of finding the best-fit curve is reduced to minimizing this sum
 \begin{equation}
   S(\alpha )\to \min. 
 \end{equation}
 
 Since no restrictions are imposed on the parameters $\alpha =(x_{0} ,y_{0} ,x_{1} ,y_{1} ,\ldots ,x_{m} ,y_{m} )$,
 over which the minimization is performed, the minimum condition is:
 
 \begin{equation}
 \frac{\partial S}{\partial x_{j} } =0,\quad \frac{\partial S}{\partial y_{j} } =0,\quad j=0,\ldots ,m.
 \end{equation}
 
 Taking into account \eqref{slabel} we obtain a system of linear equations:
 
 \begin{equation}
 \label{bsystem}
 \left\{\begin{array}{l} 
   {\sum _{j=0}^{m}(\sum _{i=0}^{n}B_{j}^{m}  (t_{i}) B_{k}^{m} (t_{i} )) x_{j} =\sum_{i=0}^{n}a_{i}  B_{k}^{m} (t_{i} ), } \\
  {\sum _{j=0}^{m}\left(\sum _{i=0}^{n}B_{j}^{m}  \left(t_{i} \right) B_{k}^{m} \left(t_{i} \right)\right) y_{j} =\sum_{i=0}^{n}b_{i}  B_{k}^{m} \left(t_{i} \right), } \\
   {k=0,\ldots ,m,} 
   \end{array}\right.
 \end{equation}
 which must be solved with respect to the variables $x_{0} ,y_{0} ,\ldots ,x_{m} ,y_{m} $.
 In fact, the system \eqref{bsystem} is split into two systems -- separately for the variables $x_{0} ,x_{1} ,\ldots ,x_{m} $
 (the first line of the system \eqref{bsystem} and for the variables $y_{0} ,y_{1} ,\ldots y_{m} $ (the second line).
 
 Let us fix the initial and final points of the Bezier curve $V_{0} =(a_{0} ,b_{0} )$ and $V_{m} =(a_{m} ,b_{m} )$,
 as well as the directions of the tangents at these points. Let us consider this problem as applied to a cubic Bezier curve.
 Let the directions at the initial and final points be given by vectors $e_{1} =(e_{1x} ,e_{1y} )$ and $e_{2} =(e_{2x} ,e_{2y} )$,
 having length 1. From the properties of Bezier curves it follows that the vertices of the characteristic polygon $V_{1} $ and $V_{2} $
 are located on the tangents at the initial and final points, respectively, i.e.
 $V_{1} =V_{0} +z_{1} e_{1} $, $V_{2} =V_{3} +z_{2} e_{2} $, where $z_{1} $ and $z_{2} $ are some constants.
 These two constants completely define the cubic Bezier curve, i.e. the vector of parameters defining the curve is $\alpha =(z_{1} ,z_{2} )$.
 The sum of the squares of the distances from the points $P=\{ P_{0} ,\ldots ,\; P_{n} \} $ to the cubic Bezier curve is
 
 \begin{eqnarray}
 S(\alpha ) &=& \sum _{i=0}^{n}\left[\left(a_{i} -\sum _{j=0}^{3}x_{j} B_{j}^{3} (t_{i} ) \right)^{2} +\left(b_{i} -\sum _{j=0}^{3}y_{j} B_{j}^{3} (t_{i} ) \right)^{2} \right] = \nonumber \\ 
 &=& \sum _{i=0}^{n}[a_{i} -x_{0} B_{0}^{3} (t_{i} ) - (x_{0} +e_{1x} z_{1} ) B_{1}^{3} (t_{i} ) - \nonumber \\
 &-& (x_{3} +e_{2x} z_{2} ) B_{2}^{3} (t_{i} )-x_{3} B_{3}^{3} (t_{i} )] ^{2} + \nonumber \\
 &+& \sum _{i=0}^{n}[b_{i} -y_{0} B_{0}^{3} (t_{i} ) -  (y_{0} +e_{1y} z_{1} ) B_{1}^{3} (t_{i} ) - \nonumber \\
 &-& (y_{3} +e_{2y} z_{2} ) B_{2}^{3} (t_{i} )-y_{3} B_{3}^{3} (t_{i} )]. 
 \end{eqnarray}
 
 From the conditions $\frac{\partial S}{\partial z_{1} } =0$, $\frac{\partial S}{\partial z_{2} } =0$ the following equations are obtained:
 
 \begin{equation}
 \label{bezierA}
 \left\{\begin{array}{l} {A_{11} \cdot z_{1} +A_{12} \cdot z_{2} =D_{1} ,} \\ {A_{21} \cdot z_{1} +A_{22} \cdot z_{2} =D_{2}, } \end{array}\right.
 \end{equation}
 where
 \begin{eqnarray*}
 A_{11} =\sum _{i=0}^{n}\left(B_{1}^{3} \left(t_{i} \right)\right)^{2}  , A_{22} =\sum _{i=0}^{n}\left(B_{2}^{3} \left(t_{i} \right)\right)^{2},
 \end{eqnarray*}
 \begin{eqnarray*}
 A_{12} =A_{21} =\left(e_{1x} \cdot e_{2x} +e_{1y} e_{2y} \right)\cdot \sum _{i=0}^{n}B_{1}^{3} \left(t_{i} \right)\cdot B_{2}^{3} \left(t_{i} \right),
 \end{eqnarray*}
 \begin{eqnarray*}
 D_{1} &=&\sum _{i=0}^{n}\left[a_{i} -x_{0} \cdot \left(B_{0}^{3} \left(t_{i} \right)+B_{1}^{3} \left(t_{i} \right)\right)-x_{3} \cdot \left(B_{2}^{3} \left(t_{i} \right)+B_{3}^{3} \left(t_{i} \right)\right)\right]\cdot e_{1x} \cdot B_{1}^{3} \left(t_{i} \right) + \\ 
 &+&\sum _{i=0}^{n}\left[b{}_{i} -y_{0} \cdot \left(B_{0}^{3} \left(t_{i} \right)+B_{1}^{3} \left(t_{i} \right)\right)-y_{3} \cdot \left(B_{2}^{3} \left(t_{i} \right)+B_{3}^{3} \left(t_{i} \right)\right)\right]\cdot e_{1y} \cdot B_{1}^{3} \left(t_{i} \right) ,
 \end{eqnarray*}
 \begin{eqnarray*}
 D_{2} &=& \sum _{i=0}^{n}\left[a_{i} -x_{0} \cdot \left(B_{0}^{3} \left(t_{i} \right)+B_{1}^{3} \left(t_{i} \right)\right)-x_{3} \cdot \left(B_{2}^{3} \left(t_{i} \right)+B_{3}^{3} \left(t_{i} \right)\right)\right] \cdot e_{2x} \cdot B_{2}^{3} \left(t_{i} \right)+ \\ 
 &+&\sum _{i=0}^{n}\left[b_{i} -y_{0} \cdot \left(B_{0}^{3} \left(t_{i} \right)+B_{1}^{3} \left(t_{i} \right)\right)+y_{3} \cdot \left(B_{2}^{3} \left(t_{i} \right)+B_{3}^{3} \left(t_{i} \right)\right)\right]\cdot e_{2y} \cdot B_{2}^{3} \left(t_{i} \right) .
 \end{eqnarray*}
 
 Thus, the solution of the system of equations \eqref{bezierA} makes it possible to find the two middle vertices
 of the characteristic quadrilateral for the cubic Bezier curve. The final coordinates of the quadrilateral vertices are as follows:
 \begin{eqnarray}
 V_{0} =(a_{0} ,b_{0} ),\quad V_{1} =(a_{0} +z_{1} \cdot e_{1x} ,b_{0} +z_{1} \cdot e_{1y} ), \nonumber \\ V_{1} =(a_{3} +z_{2} \cdot e_{2x} ,b_{3} +z_{2} \cdot e_{2y} ),\quad V_{3} =(a_{n} ,b_{n} ).
 \end{eqnarray}
 
 When approximating the boundary of a complex shape, it is usually not enough to use only one curved segment described by a Bezier curve.
 For good accuracy of approximation of a curve of a complex shape, in these cases,
 so-called composite Bezier curves are used, which are a sequential combination of several simple segments.
 In order for several simple curves to form a continuous line when successively connected,
 it is necessary to ensure that their end points coincide.
 Let $V=\left\{V_{0} ,V_{1} ,\ldots ,V_{m} \right\}$ and $V'=\left\{V'_{0} ,V'_{1} ,\ldots V'_{r} \right\}$
 be the characteristic polygons of two Bezier curves. If the end point of the first polygon coincides
 with the first point of the second $V_{m} =V'_{0} $, then these curves form a continuous line.
 If the vectors $\overline{V_{m} V_{m-1} }$ and $\overline{V'_{0} V}'_{1} $ are collinear and in different directions,
 then the connection of the curves will be smooth, i.e. at the connection point the curves will have a common tangent.
  
 In the case where the approximation of an array of points by one segment of a Bezier curve has low accuracy,
 a composite curve can be selected for which the approximation accuracy will be higher.
 
 The general structure of the algorithm for constructing a composite approximating curve looks like this.
 For definiteness, we will consider a composite cubic Bezier curve.
 
 \begin{enumerate}
\item For the initial array of points $P=\left\{P_{0} ,P_{1} ,\ldots ,P_{n} \right\}$
we construct a simple approximating Bezier curve with fixed ends.
  
\item We find the point $P_{r} $ in this array that is the farthest from the constructed curve.
In this case, the distance from the points $P_{i} ,\quad i=0,\ldots ,n$ to the curve is estimated by the value
$\Delta _{i} =\left|\sum _{j=0}^{m}V_{j} B_{j}^{m} \left(t_{i} \right)-P_{i} \right|$,
where $t_{i} $ is the value of the parameter corresponding to the point $P_{i} $, obtained in the approximation process.
 
\item If the value $\Delta _{r} =\mathop{\max }\limits_{i=0,\ldots ,n} \Delta _{i} $
satisfies us in terms of approximation accuracy, then the process is terminated.
If the deviation $\Delta _{r} $ of point $P_{r} $ from the curve is too large,
then we split the array $P=\left\{P_{0} ,\ldots ,P_{n} \right\}$ into two arrays
$P^{(1)} =\left\{P_{0} ,\ldots ,P_{r} \right\}$ and $P^{(2)} =\left\{P_{r} ,\ldots ,P{}_{n} \right\}$.
For each of these arrays, we construct our own approximating Bezier curve recursively using the same algorithm.

\end{enumerate}
 
The result of such an algorithm will be a composite Bezier curve,
in which the smoothness is violated at the points of connection of simple segments.
In order to ensure the smoothness of the composite curve, in step 3 of the algorithm,
the direction of the tangent at the point $P_{r} $ should be specified and in step 1,
an approximating curve with the specified directions at the end points should be constructed.
 
Let us define the direction of the tangents at the points of the array $P=\left\{P_{0} ,\ldots ,P_{n} \right\}$.
For the "internal" point of the array $P{}_{i} ,\quad i=1,\ldots ,n-1$, it can be defined by a vector
collinear with $\overline{P_{i-1} P_{i+1} }$. And for the "extreme" points of the array $P_{0} $ and $P_{n} $,
the direction can be chosen to be vectors tangent to the circles passing through the points
$P_{0} ,\; P_{1} ,\; P_{2} $ and $P_{n-2} ,\; P_{n-1} ,\; P_{n} $, respectively.

 \addcontentsline{toc}{section}{References}
 \begin{thebibliography}{99}
  
\bibitem{mesteckiy}
L. M. Mestetsky. Continuous morphology of binary images. FIZMATLIT 2009.

\end{thebibliography}

\end{document}